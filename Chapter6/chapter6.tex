%!TEX root = ../thesis.tex
%*******************************************************************************
%****************************** Second Chapter *********************************
%*******************************************************************************

\chapter{Summary and outlook}


\paragraph{Summary} This thesis started with an introduction to graphene and several early 2D materials such as functionalized graphene, group-IVA 2D materials and other 2D materials from layered materials, and briefly covered their diverse physical properties. Inspired by these new materials and new properties, in \autoref{chap:4} and \autoref{chap:5}, the determination of higher level physical properties of several newly discovered or proposed 2D materials were presented as the main results of the thesis. These new 2D materials are phosphorene, 2D-TiS$_3$, MXenes, 2D-Ca(OH)$_2$, pentasilicene and penta-hexa-graphene, and additionally, faceted blue PNT. Among these, the last four materials are the theoretically proposed ones from this thesis. The higher level physical properties of these materials that were investigated in this thesis include thermal, piezoelectric, magnetic and transport properties, and lithium battery-related properties. Moreover, several ways to modify the properties of these materials were introduced, such as varying the number of layers, application of mechanical strain, forming a heterostructure and the presence of defects. The main results are summarized as follows.

Phosphorene is one of the new 2D materials. It has several allotropes among which blue and black phosphorene are the most stable ones. They have band gaps around 2 eV and high carrier mobilities, which makes them ideal for electronic device application. In \autoref{thermal_phos}, we saw the highly anisotropic thermal expansion rate in black phosphorene, in contrast to the isotropic blue phosphorene. This is consistent with their structural properties. There is a less significant negative thermal expansion in these materials in comparison to graphene. Based on the free energy calculations at finite temperature, we also observed a possible phase transition from black to blue phosphorene as the temperature is increased. In this study, we combined the results from DFT for phonon calculations with the QHA to access the thermal expansion coefficient.

The piezoelectric properties of 2D-TMDs and 2D-TMDOs were evaluated in \autoref{piezo_mx2}. A number of candidates were identified to deliver better piezoelectricity as compared to the conventional bulk materials. The general trend shows the Ti,  Zr, Sn and Cr based TMDs and TMDOs have much better piezoelectric properties than Mo and W based ones. The reduction of the size and weight when using 2D materials will enhance the efficiency of electromechanical applications.

In \autoref{mag_phg}, we constructed a new type of all-carbon 2D materials through structure and bond engineering. A stable configuration can be realized such that on each of the two C atoms in the unit cell, the local magnetic moment is one Bohr magneton . This structure is a mixed composition of penta- and hexagonal C rings, and it has an AFM ground state.

Properties related to Li battery applications were studied in \autoref{Li_MG} on our proposed heterostructure, which is a bilayer of MXene and graphene. The presence of graphene breaks the stacking symmetry, such that Li atoms bind stronger to MXenes and stay far away from the graphene layer. The binding energy is larger than that in multilayer MXenes, while the diffusion barriers in these two cases are comparable. 

2D-Ca(OH)$_2$ is another 2D material proposed in this thesis. In \autoref{CaOH2_layers}, we have seen negligible influence of the dimension of these materials, despite having a hydrogen-enhanced interlayer interaction. Interestingly, a free-electron-like planar surface state exists in the monolayer and this state has a symmetric charge density distribution with respect to the layer. This symmetry is broken when the system is transformed to a bilayer.

Pentasilicene is a variation of pentagraphene in which carbon atoms are replaced with silicon ones. However, it is not stable as has been indicated by the phonon dispersion having imaginary frequencies. In \autoref{pSi_layers}, we stabilized it by transforming it into a bilayer structure where the interlayer bonds are covalent. Surprisingly, the resulting structure is the most stable bilayer silicene to the best of our knowledge.

Strain can also be used to alter the properties of materials. In \autoref{mob_Tis3}, we used it to enhance the acoustic-phonon limited mobility in 2D-TiS$_2$. From the deformation potential theory, we explained the proportional relationship of the DPC with the band edge shifts with respect to vacuum under the application of homogeneous strain. The latter is small at a finite strain, which implies the mobility of the material can be large when strain is applied. In this way, more than an order of magnitude increase of the mobility was achieved for this material.

Another way to modify the properties of 2D materials is to construct a heterostructure. In \autoref{trans_mx2}, we studied the electrical transport properties of such a system. Taking advantage of the coherent interface of the T and H phase of MoS$_2$, we formed a lateral heterostructure where two metallic T phases form the electrodes and one semiconducting H phase is the scattering center. The transmission was studied under various electron concentrations to reduce the Schottky barrier height which has been the main obstacle for electron transport at the metal-semiconducting interface. A n-type doping at one electrode and a p-type at the other can reduce the barrier height for both electron and hole. 

Finally, a new type of faceted nanotube was designed from blue phosphorene in \autoref{defect_phos}. By introducing line defects in the 2D plane, kinks can be formed to close a faceted tube surface. The advantage of this faceted tube is the fact that flat edges almost do not cost elastic energy to form. This is energetically more favourable than a curved surface of conventional nanotubes. The most important part of the structure are the corners of the faceted tubes, which determines the energy and the electronic band gap of the tubes.

\paragraph{Outlook} Nowadays, more and more stable 2D materials are being proposed and synthesized. They need to be explored to fully reveal their potential in both applications and fundamental research. For example, the MXenes family has more than 60 members with a broad range of properties waiting to be characterized. Moreover, with the increase of the number of different 2D materials increasing, the possible combinations of forming heterostructures exhibit a factorial increase. In addition, these heterostructures do not have to be limited to 2D, if one stacks them to a macroscopic thickness, one will end up with layered bulk materials, or nanocomposites, with tailored properties that are designed at the microscale. On the other hand, DFT software packages are being optimized for higher efficiency and broader capabilities to take full advantage of modern supercomputing facilities. Therefore, the determination of high-level properties of materials of larger simulated systems become accessible and will overcome the barriers that previously limited our understanding of new 2D materials and opens the door to use them in new applications.