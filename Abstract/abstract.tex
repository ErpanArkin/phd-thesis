% ************************** Thesis Abstract *****************************
% Use `abstract' as an option in the document class to print only the titlepage and the abstract.
\begin{abstract}

This thesis is a collection of works aimed at exploring the physical properties of novel two-dimensional (2D) materials using advanced first-principles calculations in the framework of density functional theory (DFT). 2D materials have an atomic thickness. Because of this reduction of the dimensions, quantum confinement effects will have an important influence on the properties of the material. The field of 2D materials is expanding with the continued introduction of new members and the new phenomena they bring. DFT is a powerful tool that can be used to probe and measure these exciting materials and their properties. 

By performing advanced DFT calculations, this thesis proposes several new 2D materials. These materials are stable and have distinct properties as compared to their bulk form. The discovery of new 2D materials involves separating known layered materials in their bulk form into monolayers or rationally designing materials with tailored properties at the atomic level. Full characterizations are given to introduce these new materials to the 2D materials community. 

In addition to the basic properties of these materials that can be obtained by a single step DFT calculation, this thesis increases the level of the characterization by introducing additional approximations on top of the DFT calculations. This usually requires several calculations to be done for different conditions, e.g. mechanical strains, to find the relation between these results. By doing so, the properties related to more dynamical processes of the materials can be understood.

This thesis also introduces several effective ways to modify the determined properties. Questions like "What is the effect of the interlayer interaction for a few-layer system" and "How a particular material responds to the application of strain?" are answered throughout the thesis.


\end{abstract}
