% ************************** Thesis Abstract *****************************
% Use `abstract' as an option in the document class to print only the titlepage and the abstract.
\begin{abstract}

This thesis is a collection of works aim to explore the physical properties of novel two-dimensional (2D) materials using advanced first-principles calculations in the framework of density functional theory (DFT). 2D materials are the type of materials having an atomic thickness. Because of this reduction of the dimensions, the effect of quantum confinement will have an important role to alter the properties of the material. The field of 2D materials is expanding with the introduction of new members and the new phenomenon they bring. DFT is a powerful tool can be used to probe and measure these exciting materials and their properties. 

By performing advanced DFT calculations, this thesis proposes several new 2D materials. These materials are stable and have distinct properties as compared to their bulk form. The invention process involves separating known layered materials in their bulk form into monolayers and rational designing material with tailored properties at the atomic level. Full characterizations are made to introduce these new materials to the 2D materials community. 

In addition to the basic properties of the materials that can be determined by a single step DFT calculation, this thesis expands the slope of the characterization to a higher level by introducing addition approximations on top of DFT calculations. This will usually need several calculations to be done for different conditions, e.g. mechanical strains, and find the relation between these results. By doing so, the properties related to the dynamical process of the materials can be understood.

This thesis also introduces several effective ways to modify the determined properties. Questions like "What is the effect of the interlayer interaction for a few-layer system" and "How a particular material responds to the application of strain?" are answered throughout the thesis.


\end{abstract}
